\addcontentsline{toc}{chapter}{\protect Zusammenfassung}
\markboth{Zusammenfassung}{Zusammenfassung}
 
\chapter*{Zusammenfassung}
Die Wechselwirkung zwischen dem zentralen Schwarzen Loch einer Galaxie und den Baryonen (Gas, Staub und Sterne) innerhalb der galaktischen Scheibe hat einen entscheidenden Einfluss auf die Entwicklung der Galaxie. In dieser Arbeit soll der Einfluss von Akkretion eines Schwarzen Lochs auf die Aktivit\"at des aktiven galaktischen Kerns (AGN) demonstriert werden. Zuerst wird durch einen statistischen Test zwischen den 1.4 GHz Leuchtst\"arken und den pseudo-3D Dichtemessungen von 2716 Radio AGN, welche mit dem Karl G. Jansky Very Large Array (JVLA) beobachtet wurden, mit $z\lesssim0.8$ gezeigt, dass die Leuchtkraft eines kosmischen Jet und die Galaxienanzahldichte der Umgebung weitestgehend unkorreliert sind. Stattdessen ist es wahrscheinlicher, dass interne Prozesse, wie etwa die Akkretion von kaltem Gas auf einen AGN, eine weitaus wichtigere Rolle als Regulator der AGN Aktivit\"at spielen. Als Zweites werden Linien mit Ruhewellenl\"angen im UV Bereich, gemessen im Zentrum der $z=2.92$ Radiogalaxie MRC 0943-242 mit dem Very Large Telescope (VLT) Integralfeldspektrograph MUSE (Multi-unit Spectroscopic Explorer), dazu benutzt, die Emissions- und Absorptionskoeffizienten des ionisierten Gas von Gaswolken in der n\"aheren Umgebung der Galaxie zu bestimmen. Die Beobachtung von blauverschobenen Emissionslinien relativ zur systematischen Geschwindigkeit der Galaxie weist auf Radiojetgetriebene Gasausstr\"omungen hin. Absorptionslinien suggerieren, dass sich eine gro\ss skalige ($r\,\gtrsim\,60$\,kpc), mit Metallen angereicherte H\"ulle von absorbierendem Gas jenseits der Bugsto\ss welle des Radiojets befindet. Die Position dieses absorbierenden Materials suggeriert, dass es durch eine fr\"uhzeitig einsetzende AGN R\"uckkopplung gebildet wurde. Abschlie\ss end wird ein Pilotprojekt pr\"asentiert, welches die mit ALMA (Atacama Large Milimeter-submilimeter Array) f\"ur sieben Galaxien mit hoher Rotverschiebung (HzRGs, $z\gtrsim2.9$) gemessene [C\,\textsc{i}](1-0) Feinstrukturlinie als Indikator f\"ur molekuklares Gas nutzt. Die [C\,\textsc{i}](1-0) Linie wird alternativ oft als Indikator f\"ur molekularen Wasserstoff (H$_2$) mit niedriger Metallizit\"at und Dichte genutzt. Das Ergebnis dieses Studie zeigt, dass f\"ur sechs der sieben HzRGs die [C\,\textsc{i}](1-0) Feinstrukturlinie nicht gemessen werden konnte. F\"ur die Galaxie TN\,J0121+1320 mit $z=3.52$ konnte die [C\,\textsc{i}](1-0) Linie jedoch mit einer Halbwertsbreite von $\approx160$\,km\,s$^{-1}$, verbreitert durch AGN Jet-getriebene Turbulenzen des molekularen Gases im interstellaren Mediums (ISM), nachgewiesen werden. Eine blinde Suche nach [C\,\textsc{i}](1-0) in der Umgebung der Galaxien resultiert in engen Linien mit einer Halbwertsbreite zwischen 20 -- 100\,km\,s$^{-1}$ in einem projezierten Abstand $d\geq10$\,kpc vom galaktischen Zentrum, indikativ f\"ur kinematisch ruhiges molekulares Gas im zirkumgalaktischen Medium (CGM). Das Fehlen von H$_2$ Gas in sechs von sieben Galaxien bedeutet, dass fast der gesamte Vorrat an molekularem Gas schon durch fr\"uhere Phasen rascher Sternentstehung aufgebraucht wurde. Diese Studien zeigen, dass die Akkretion von Material auf Schwarze L\"ocher und die Energieabgabe von Radio-Jets eine hohe Bedeutung f\"ur die Kinematik und Morphologie des Gases im ISM und CGM von Ragiogalaxien haben.