\addcontentsline{toc}{chapter}{\protect Abstract}
\markboth{Abstract}{Abstract}

\chapter*{Abstract}

The interaction between a galaxy's central black-hole and the baryons (gas, dust and stars) within the host galaxy's disk plays a crucial role in influencing a galaxy's evolution. In this work, the influence of black-hole accretion on active galactic nucleus (AGN) activity is demonstrated. This is {\it initially} done using a statistical test between the 1.4 GHz luminosities, and pseudo-3D density measures of 2716 radio AGN detected by the Karl G. Jansky Very Large Array (JVLA) at $z \lesssim 0.8,$ which shows that the jet power of the AGN, traced by the radio luminosities, and the galaxy number density are largely uncorrelated. Rather, secular processes in the form of cold gas accretion onto the AGN are likely to play a more important role in regulating AGN activity. {\it Secondly,} by fitting the rest-frame ultraviolet (UV) spectral lines detected by the Very Large Telescope (VLT) integral field unit spectrograph MUSE (Multi-unit Spectroscopic Explorer), we obtain detections of the nuclear region of the $z=2.92$ radio galaxy, MRC 0943-242. On the line and continuum spectra extracted from this region, we parametrise the properties of gas within the galaxy halo in terms of its line absorption and emission. By fitting the emission lines, we obtain evidence of gas blueshifted up to $\sim$1000 km s$^{-1}$ relative to the galaxy's systemic velocity, suggesting radio jet-driven bulk outflows of gas. The absorption-line fits indicate the presence of a large-scale ($r \gtrsim 60$ kpc), metal-enriched shell of absorbing gas located beyond the bow-shock of the radio jets. 
The location of this absorber suggests that it was formed by an early AGN feedback event. {\it Finally,} we present a pilot study of molecular gas traced by [\ion{C}{i}](1-0) detected with ALMA (Atacama Large Milimeter-submilimeter Array) in seven high-redshift radio galaxies (HzRGs) at $z \gtrsim 2.9.$ The [\ion{C}{i}](1-0) fine-structure line is often used as an alternative probe for molecular hydrogen, H$_2,$ under low gas density and metallicity conditions. The results show that six of the seven HzRGs in the sample have non-detections of [\ion{C}{i}](1-0). The galaxy TN J0121+1320 at $z=3.52,$ however, has a clear [\ion{C}{i}](1-0) detection of line-width of $\rm FWHM\simeq160$ km s$^{-1}$ broadened by AGN jet-driven turbulence of molecular gas within the interstellar medium (ISM). A blind search for [\ion{C}{i}](1-0) in the field surrounding the galaxies results in detections at projected distances $d \gtrsim 10$ kpc from the hosts which have line-widths of $\rm FWHM = 20 - 100$ km s$^{-1}$ indicative of kinematically quiet molecular gas in the circumgalactic medium (CGM). The fact that H$_2$ gas is not traced in six of the host galaxies in the sample implies that large reservoirs of molecular gas in the host galaxy ISMs were consumed through earlier phases of star-formation. These studies demonstrate the sheer importance of the black-hole-powered AGN and radio jets in altering the kinematics and morphology of gas within the ISM and CGM of a radio galaxy. 