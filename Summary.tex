\chapter[Summary]{Summary}

Overall, this aim of this thesis has been to demonstrate that the central black-hole of a galaxy is an important driver of radio active galactic nuclei (AGN) activity as well as the kinematics and morphology of gas surrounding it within both the interstellar medium (ISM) and circumgalactic medium (CGM) surrounding a radio galaxy. Indeed, throughout this the thesis, I have provided evidence in support of this for low-$z$ ($z \leq 0.8$) radio-selected AGN as well as for high-redshift radio galaxies (HzRGs) at $2.9 \lesssim z \lesssim 4.6$ which we have combined I discuss, here, the implications of the main findings and their conclusions within the broader context of galaxy evolution. I also highlight further questions that arise from these studies and how they may be addressed in future with new developments in instrumentation and computing capability for simulations. \\

\subsection{Cold-mode accretion in radio AGN at $z \lesssim 1$}

In the second chapter, I presented a statistical study of J-VLA (Karl G. Jansky Very Large Array) radio-selected AGN in the Stripe 82 field. The results showed that no correlation could be found between radio power traced by 1.4 GHz luminosities and close group environment traced using a pseudo-3D density measure to the 2$^{\rm nd}$ and 5$^{\rm th}$ nearest neighbours. In the absence of a correlation, I deduced that cold accretion flows onto the AGN was perhaps the major influencer radio AGN activity measured. Furthermore, the radio source group environment densities exceeded those of optical and near-infared selected field galaxies by a significant margin. If secular cold-mode accretion is, indeed the main driver of AGN activity, how can we use observe this directly for $z \lesssim 1$ AGN? Does cold accretion drive AGN power at redshifts, $z > 1$?\\

Simulations of Bondi as well as chaotic cold mode accretion have provided a basis on what observations black-hole accretion lead to (\citet{Gaspari2013} and references therein). In particular, provided there is a sufficiently bright emission background provided by a radio AGN, for example, absorption lines can be a reliable method for probing inflows (as well as outflows) of gas in the cold phase. This can be traced by the
redshift and blueshift of lines tracing foreground molecular gas probed by CO (carbon monoxide), SiO (silicon oxide), cyanide (CN), atomic carbon (CI) among other tracers \citep{David2014,Tremblay2016,Ruffa2019,Rose2019}. Without a doubt, milimeter/sub-milimeter interferometers such as ALMA (Atacama Large Milimetre Array), APEX (Atacama Pathfinder Experiment), PdBI (Plateau de Bure Interferometer) and others of this calibre will continue to be fundamental in obtaining observations of molecular absorption. Most of this work has been carried at for very low-$z$ sources ($z \lesssim 0.1$). Intuitively, higher sensitivity interferometers, will be required to trace this infall of molecular gas clouds within galaxy cores. \\

\subsection{Enrichment and ionisation of large-scale absorbers within the CGMs of HzRGs}

The case study of the $z=2.92$ radio galaxy, MRC 0943-242, presented in the third chapter confirmed the presence of an neutral hydrogen, HI, absorbing gas shell with a high column density of $N_{\rm HI}/\rm{cm}^{-2} \simeq 10^{19}$ based off of aborption line fitting to the Ly$\alpha$ $\lambda$1216 The absorber is blueshifted relative to the bright emission line region of the galaxy thus foregrounded and possibly outflowing. We have been able to demonstrate that CIV and NV absorption coincides in line-of-sight velocity with the HI absorber implying that it is metal-enriched and not merely composed of neutral hydrogen. \\

The result of the study appeared to be somewhat paradoxical with the comparison between HI, CIV and NV column densities measured for this enriched absorber and {\it Cloudy} photoionisation models \citep{Ferland2013}. The comparisons suggested that the AGN (with a spectral energy distribution that follows a power-law) photoionises the absorber which has a metallicity of Z/Z$_\odot \simeq 0.01$ and a nitrogen abundance a factor of 10 greater than usual at this metallicity i.e. [N/H] = 10. Why is there a nitrogen over-abundance? Is metal enrichment through starburst-driven superwinds the reason for the observed abundances? To improve this study and others of its kind, I would suggest having: (a) a tesselation-based map of the HI column density from the IFU datacube, (b) a Bayesian Monte-Carlo Markov Chain (MCMC) approach for fitting the multivariate emission and absorption functions and, (c) a more detailed photoionisation model that accounts for the AGN photoionisation and stellar enrichment in a more sophisticated rather than simplistic manner. \\

\subsection{Molecular hydrogen reservoirs within the CGMs of HzRGs}

The fourth chapter of this thesis is a pilot study of [CI](1-0) line emission traced within the ISMs and CGMs of seven HzRGs. Measuring the line widths, velocities and flux densities of the [CI] line emission, we were able to measure infer the kinematics, morphology and H$_2$ masses, $M_{\rm H_2}$ within the molecular gas reservoirs. Our findings showed that at the locations of the host galaxies, $M_{\rm H_2}/\rm{M}_\odot \simeq 10^{10}$ and line widths of of the order of $\rm FWHM \simeq 160$ km s$^{-1}$ (for one of the HzRGs). Detections of [CI] line emission further out within the gas haloes (in the CGMs) have H$_2$ gas clouds that are less massive, $M_{H_2}/\rm{M}_\odot \simeq 10^9$ and line-widths indicating more quiescent kinematics, $\rm FWHM=20 - 100$ km s$^{-1}.$ The conclusion drawn from this is that, once more, mechanical jet power is responsible for the turbulent motion of molecular gas within the ISMs of the HzRGs. \\

This study could be improved by adding $^{12}$CO(1-0) line detections ($\nu_{\rm rest} =115.3$ GHz) for the seven galaxies in the sample examined. The interferometer, ATCA (Australia Telescope Compact Array) or ALMA have the bandwidth and spectral capability to detect this line at the redshift range of the sources. With the addition of CO, the ionisation conditions within the molecular clouds can be better traced \citep{Gullberg2016b,Emonts2018,Papadopoulos2018}, even more so with a comparison of the [CI]/CO line ratios to the output models from photodissociation region (PDR) codes \citep{Bothwell2017}. 