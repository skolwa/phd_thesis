\documentclass[11pt,a4paper,draft]{article}
\usepackage[utf8]{inputenc}
\usepackage{amsmath,amssymb,amsfonts}
\usepackage{natbib}
\textwidth 420pt
\parindent 0pt
\oddsidemargin 20pt
\usepackage{url,hyperref}

% journal shorthand
\newcommand\aj{AJ}                   	% Astronomical Journal (the)
\newcommand\mnras{MNRAS}    % Monthly Notices of the Royal Astronomical Society
\newcommand\aap{A\&A}              % Astronomy and Astrophysics
\newcommand\apjs{ApJS}               % Astrophysical Journal, Supplement
\newcommand\apj{ApJ}                 % Astrophysical Journal
\newcommand\nat{Nature}              % Nature
\newcommand\araa{ARA\&A}             % Annual Review of Astronomy and Astrophysics

\begin{document}

\title{{\bf DRAFT 2:} section 1.1 - Feedback Mechanisms in Radio Galaxies} 
\author{S. Kolwa}
\maketitle

A plethora of evidence now exists in support of the notion that the central supermassive black-hole (BH) of a galaxy can have a significant impact on the evolution of the galaxy hosting it. In particular, the well-constrained empirical relation between the central black-hole mass (M$_\bullet$) and the stellar velocity dispersion ($\sigma$) of a galaxy is evidence of this fact \citep{KormendyHo2013}. This M$_\bullet$-$\sigma$ relation is important in establishing the importance of the black-hole, a relatively small-scale object at the centre of a galaxy, with the larger distribution of baryons (stars, gas, and dust) surrounding it.

The central black-hole can indeed have a dramatic effect on a galaxy's measurable properties when it is sufficiently massive, in other words, when it has a mass ($M_{\rm BH}$) well above a threshold of approximately $M_\bullet \gtrsim 10^8 - 10^9~\rm M_\odot.$ In this case, the central regions of a galaxy become, what is known as, an active galactic nucleus (AGN). The influence that the AGN has on the physics of the baryons within and around the galaxy are described as AGN feedback mechanisms. They can impact the evolution of a galaxy through large expulsions of energy approximately equivalent to the black-hole rest energy ($E_\bullet$), $E_\bullet = \epsilon M_\bullet \rm{c}^2$ where $\epsilon$ is the efficiency of the energy output and $c,$ the light-speed. This energy output can consequently impact the availability of molecular gas reservoirs from which stars form in a galaxy and as a result affect a galaxy's further growth. AGN feedback can also drive out gas through outflows, for instance, which would result in star-formation within the galaxy slowing or ceasing completely (negative feedback). The energy output of the AGN, whether radiative or kinetic, may also enhance the star-formation rate through shock-driven collapse of giant molecular clouds, for example -- this would be an example of positive feedback. These are rather simplified scenarios, however, that do not take into account other physical processes operating in tandem with AGN feedback. 

Galaxies with very high radio luminosities, radio galaxies, experience the influence of AGN feedback. In radio galaxies, which this thesis focusses primarily on, AGN feedback operates in the form of powerful energy injections that can affect the thermodynamics, kinematics, morphology, ionisation state, and displacement of gas that is gravitationally bound to a galaxy within the interstellar and circumgalactic mediums (ISM and CGM). Since the rate of mass accretion onto the central black-hole determines its consequent rate of its energy output given by the total luminosity of the black-hole ($L_\bullet$), using a time derivative of the rest-mass equation the mass accretion rate ($\dot{M}_\bullet$) is proportional $L_\bullet$: $\dot{M}_{\rm acc.} = \epsilon L_\bullet c^{2}.$ 

Radio galaxies are a main focus in this thesis and, within them, AGN feedback operates through the mechanical power of the radio jets that are produced by non-thermal, synchrotron radiation from the accretion disk surrounding the black-hole. This form of feedback is the kinetic-mode and is radiatively inefficient unlike the radiative-mode of feedback for which the energy output is in the form of radiation (or photon winds) hence the feedback is bimodal \citep{HeckmanBest2014}. Since jet outflows that occur in radio galaxies are responsible for the kinetic-mode of feedback, I direct my focus to this form of feedback. Kinetic feedback is well studied at low redshifts ($z \lesssim 1$) hence many of these resolved radio galaxies are considered evolutionary analogues to the radio galaxies at higher redshifts. 

In thesis, I use a statistical study of the influence of environment density and radio luminosity to determine whether low redshift ($z \leq 0.8$) sources in groups are more likely to host powerful radio AGN. We extend the study of radio source environments to the high-redshift universe by studying a sample of seven radio galaxies at $z \gtrsim 2.9,$ with the aim of determining the impact of their powerful radio jets on their ISM and CGMs. Through observations, we can determine the effect of radio jets on the ionised, molecular and neutral phases of the gas. This is crucial to understanding the ways in which kinetic-mode feedback can alter the consequent evolution of a galaxy \citep{Fabian2012}. Observing this in high redshift radio galaxies (HzRGs), in particular, are essential to unlocking the mystery behind AGN feedback within the early Universe, 1-3 Gyr after the Big Bang. 

\newpage
\bibliographystyle{mnras}
\bibliography{intro_1_1}

\end{document}